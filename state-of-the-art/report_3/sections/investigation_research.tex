\section{Investigation/Research}

In this section we describe our investigation and research over the proposal readings. Investigation throughout the examination of the reading facts, novel possibilities and results. Also, addressing the research reasoned conclusions of those readings.

\subsection{HRS Perspectives}

Due to the specificity of the criteria benchmarks in clinical scenarios, Recommender Systems (RS) have not been used yet extensively in Health Informatics (HI) and Medical Imaging (MI), despite of being a classical example of Machine Learning (ML) applications. Also, the multitude of the clinical domain is drastically differing the groups of the end-user and the large context complexity of the domain. By using a Human-In-The-Loop (HITL) approach, that for our work we will use the word Clinician-In-The-Loop (CITL), since we are applying it to the clinical domain, several difficulties could be mitigated by combining both human expertise with computer efficiency. These last difficulties and opportunities were already addressed on \href{https://github.com/mida-project/reading-reports/blob/master/state-of-the-art/report_1/main.pdf}{Report 1} and \href{https://github.com/mida-project/reading-reports/blob/master/state-of-the-art/report_2/main.pdf}{Report 2}.

For these needs, a work done by Valdez et al. \cite{valdez2016recommender} with the title \textit{Recommender Systems for Health Informatics: State-of-the-Art and Future Perspectives}, provides a research framework to access HRS. The authors are suggesting to incorporate a health domain understanding, evaluation and specific methodology into the development process. Since the fact that our \textbf{Medical Imaging Diagnosis Assistant (MIDA)} can be integrated as a HRS also, it is of chief importance to undress this kind of works.

While it is considered the whole HRS in the real-world scenarios, it is mandatory to understand how users, in our case clinicians, will interact with a medical assistant. Therefore, it is important the application of an User-Centered Design (UCD) approach, or more specifically, a Clinician-Centered Design (CCD) framework. depending on domain knowledge, different types of interactions are most supportive to clinicians. Top-recommendations are preferred by novice clinicians, while expert clinicians prefer hybrid-approaches combining explicit and implicit preferences.

Even though HCI discipline has become an increasingly important variable of the HRS, it is still a tiny part in research overall \cite{calero2016hci}. Moreover, aspects like medical domain, has not had a large share of research. So that, it is more and more evident that we must address the problem/solutions of HI supported by HCI. These authors framework merely looks at the challenge of HCI over HI in a perspective of extending it into the areas of Information Visualisation (InfoVis) and technology acceptance. Both are needs of our \textbf{MIDA} project and research.

\subsection{HRS Awareness}

The HRS Awareness is covered by a short paper with the title \textit{Towards Health (Aware) Recommender Systems} \cite{schafer2017towards}, where the authors are looking for one step ahead and are showing the progress made towards HRS helping clinicians. The authors are analysing the complex medical interventions, supporting clinicians with preventive healthcare measures. Also, in this work, it is identified the key challenges that need to be addresses for an effective HRS, where the kind of decision support needs in this clinical domain characterised by being of a high-risk domain.

At the end the authors argued that a definition of multidimensional user satisfaction will feed the HRS needs for being evaluated. By testing those in a real-world situation, it will prevent the harmful, unethical, behaviours using expert guidance and fallbacks. Once challenges are transcended, it will be possible to evolve sophisticated systems such as our assistant.

\subsection{HRS Challenges and Opportunities}

Several challenges and opportunities of merging HRS are discussing \cite{fernandez2009challenges} the usage of semantic technologies within the above approach. User models are containing the parameters predefining the metal status and expertise of the clinicians. The initial knowledge about the user of HRS is limited. As a matter of fact, the use of extensive questionnaires regarding above parameters, is not common since it might hinder an interest in using the system from clinicians. Normally, HRS is not having access to any kind of datasets where typically knowledge is gathered from analysis of interactions.

On one hand, the approach is weak in the time needed to establish the user model, but on the other hand, the approach is strong in automatising the user model. In our \textbf{MIDA} Project, the datasets are created by medical experts. To incorporate new diagnostic images it has to be described by the medical experts and may imply to adjust the new learning algorithm. Although it makes it difficult to dynamically incorporate additional resources, this approach guarantees the quality of those resources.

The challenges described by the authors are therefore important to us, while we need to understand several issues we face through the research process. One of the challenges were the questionnaires, where it can be a burden to the patients. Authors access to the information of the users and reduced the problem of the lack of information about new users. They also face integration problems. The user’s analysis could enrich the traditional approach by taking into account the user's preferences. Another important information to take into consideration to our work. For the opportunities on their work,  the authors argue that platforms are used in HRS, not just to generate datasets but also to harvest knowledge about clinicians and information. By using a HRS the Medical Imaging analysis will be used to discover automatically lesions on the images, guiding the clinicians on the interpretation and resolution.