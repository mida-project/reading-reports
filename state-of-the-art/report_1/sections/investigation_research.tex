\section{Investigation/Research}

In this section we describe our investigation and research over the proposal readings. Investigation throughout the examination of the reading facts, novel possibilities and results. Also, addressing the research reasoned conclusions of those readings.

\subsection{RSVP}

Rapid Serial Visual Presentation (RSVP) is frequently used as an experimental model to examine the temporal characteristics of attention. A work done by Brown at al. \cite{brown2017role} states a riffling the pages of a book using these techniques. By using a set of images per second, showing evidence of an often possible successful search.

The authors tested a comparable RSVP designs in two different illusions, a "Deep-Flat" visual illusion and a 2D Flat plane. Both, ascertain the relative effects of 2D and 3D under precisely controlled conditions as style presentation. Moreover, in this study, an elicited data includes the performance measures, user preferences and opinions of the participants. At the end, the authors established an effectiveness RSVP by using the two illusions.

This work could be important to us since it shows different 'modes' of RSVP as a rapid sequential presentation that could be applied to the visualisation of our medical images. Therefore, Medical Imaging, and more specifically our system, can take advantage of this RSVP 'modes', since it can be especially helpful if, after the appearance of several DICOM \cite{mildenberger2002introduction, pianykh2009digital} series with a set of instances (each) from various studies belonging to one patient, it can remain in view to allow the radiologist to confirm that a desired instance as been found.

For the experimental procedure the authors had a total of 25 participants showing them a list of five possible categories. All participants did an on-screen questionnaire ate the end of each sequence presentation. The purpose of this questionnaires was to elicit aspects of participants' overall experience. These questionnaires were answer on a five point Likert scale. Moreover, participants were asked at the end about fatigue and preference between the different interfaces. For the results and analysis, the authors the Kruskal-Wallis rank sum test \cite{theodorsson1986kruskal} to understand the significance of the comparable conditions. Where if a comparison result is taken as representative, researchers should employ perspective cues with caution. Finally, for a user opinion study, the authors used the Wilcoxon test \cite{wilcoxon1945individual}, a pairwise ANOVA \cite{hoaglin1978hat}, to indicate design combinations and to found statistical significance under user opinions.

\clearpage

At the end, the authors concluded that performance was not significantly affected by the illusion of depth, when tested under directly comparable conditions. However, the inclusion of a certain background cues can have a significantly detrimental effect on performance. That said, it is important to take conclusions into consideration for our work. While medical images have typically a black background and are surrounded by the viewport tools and features. Also typically black. Changes of the design could, in fact, rise to significant improvement in recognition performance. There is evidence that combinations of visual depth cues substantially and significantly reinforces the visual effect. Based on this evidence, authors encourage the exploration of a larger design alternatives and applications to further research of the effects of these alternatives. Therefore, it is an interesting proposal to test several alternatives on a Medical Imaging application by testing task performance and radiologist's workflow changes.

\subsection{Labeling Images}

At \textit{Labeling Images with a Computer Game} paper, Luis von Ahn et al. \cite{von2004labeling} introduces a new interacting system that can be used to create valuable output. The idea of the authors is to put people playing a game that will help determine the contents of the images by providing meaningful labels for them. Their system makes a significant contribution to us since its valuable output of generated dataset and the way it addresses the image-labelling problem. As we also need to 'label' (annotations) each medical image. Authors encourage people to do the work by entertaining them, while having their desire as an advantage, rather than using Computer Vision (CV) techniques. The last ones, do not work well enough.

This game and the authors work are both a novel interactive system that allows people to label images while enjoying themselves. The authors have showed evidence that people will play their game and that the labels it produces are meaningful. The hereby work, can be complementary to our work since it shows a way to motivate users to label several images. A similar task and goal that we have in our work. In our work we have radiologists as users. As many as we can have, and as more expert they are, better and larger dataset we will have. Having proper annotations associated to each medical image on a more scalable environment, like the web, will give us the advantage of having more radiologists at the same time. Also having international experts on radiology. It will allow the system to have a more accurate dataset, with a more accurate on severe cases, improving the diagnostic assistance supported by automatic system.

\subsection{Interactive Machine Learning}

In this section we describe the analysis to an Interactive Machine Learning (iML) work for Health Informatics (HI), where Andreas Holzinger \cite{holzinger2016interactive} defines iML as \textit{"algorithms that can interact with agents and can optimize their learning behaviour through these interactions, where the agents can also be human"}. But first we need to address the goal of Machine Learning (ML), while being the development of algorithms which can learn and improve over time, used to predictions. Most of the researchers are concentrate on Automatic Machine Learning (aML) which bring us a great advantage as an automatic approach benefits from big data with many training sets. But this bring us a disadvantage, since in the health domain, we are in presence of a little number of events or available people to generated those datasets, where approaches like aML suffer of insufficient training samples. Here, the author brings a solution to this. The iML is therefore a viable approach for HI, having its roots in Active Learning (AL) and Reenforcement Learning (RL).

\clearpage

The AL idea is giving a more powerful learning process pairwise with the RL technique, achieving greater accuracy with fewer information, if it is allowed to choose the data from which it learns. A system with this approach may pose queries to be annotated by a radiologist. AL is well motivated through us, where our non-annotated medical images may be abundant or less easily obtained, but radiologists annotations are expensive, time-consuming or difficult to obtain.

Summarising, the authors strongly recommends an emphasising application with success of Machine Learning (ML) over Health Informatics (HI). Requiring a concerted effort of researchers from different areas. Thereby, combining human cognition with iML approaches can be of particular interest to solve problems in HI, while it has a lack of large datasets.

\subsection{Volume Segmentation}

Igarashi et al. \cite{igarashi2016interactive} presented an interactive method for segmentation and isosurface extraction of a medical volume data. The authors are proposing an alternative approach whereby users simply applies painting operations to the volumes using commonly seen in painting system tools, such as brushes and flood fill. Motivated for this work, while in conventional methods, users are aim to decompose a volume into multiple regions iteratively. Using a threshold by each segmentation region, users manually clean the segmentation result by removing clutter in each region. Therefore, it is proposed a way to solve this tedious issue while it requires many mouse operations from different views. By significantly reduces the number of camera and mouse operations, their technique contribution is in the introduction of the threshold field. It assigns spatially-varying threshold values to individual voxels. The paper, also describes the user interaction details and implementation. Generalising the discrete decomposition of a volume into regions. Thereby, offering a much more efficient and flexible workflow, by segmentation using a constant threshold in each region. The several results made on this study are indicating that the proposed method can be faster than a conventional method. While this work is an introduction of new interaction techniques giving us knowledge regarding how to develop and study novel techniques that will improve clinical systems.

In conclusion, the authors have presented an interactive method for volume segmentation. Their contribution to our work is the introduction of a study where volume segmentation and threshold field are underlying our data visualisation. The authors also showed us an analytical study for their target users, that in their samples are neurosurgeons, similar to our radiologists profile. The users found their proposed method highly novel, expressing a strong desire to use in a real-world practice. Despite of a small-scale of four users, the paper shows that the proposed method can outperform a conventional method. The contribution in the medical imaging fields is orthogonal to optimisation techniques, while we can combine them to facilitate advanced annotation process. Not forgetting the fact that our radiologists are spending many hours in medical imaging observation and diagnosis, thus their methods, can make a significant impact in our system, while it can greatly increase the speed at which this tasks can be completed. Therefore, reducing the radiologist's workflow.