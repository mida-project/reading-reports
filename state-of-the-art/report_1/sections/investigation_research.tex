\section{Investigation/Research}

In this section we describe our investigation and research over the proposal readings. Investigation throughout the examination of the reading facts, novel possibilities and results. Also, addressing the research reasoned conclusions of those readings.

\subsection{RSVP}

Rapid Serial Visual Presentation (RSVP) is frequently used as an experimental model to examine the temporal characteristics of attention. A work done by Brown at al. \cite{brown2017role} states a riffling the pages of a book using these techniques. By using a set of images per second, showing evidence of an often possible successful search.

The authors tested a comparable RSVP designs in two different illusions, a "Deep-Flat" visual illusion and a 2D Flat plane. Both, ascertain the relative effects of 2D and 3D under precisely controlled conditions as style presentation. Moreover, in this study, an elicited data includes the performance measures, user preferences and opinions of the participants. At the end, the authors established an effectiveness RSVP by using the two illusions.

This work could be important to us since it shows different 'modes' of RSVP as a rapid sequential presentation that could be applied to the visualisation of our medical images. Therefore, Medical Imaging, and more specifically our system, can take advantage of this RSVP 'modes', since it can be especially helpful if, after the appearance of several DICOM \cite{mildenberger2002introduction, pianykh2009digital} series with a set of instances (each) from various studies belonging to one patient, it can remain in view to allow the radiologist to confirm that a desired instance as been found.

For the experimental procedure the authors had a total of 25 participants showing them a list of five possible categories. All participants did an on-screen questionnaire ate the end of each sequence presentation. The purpose of this questionnaires was to elicit aspects of participants' overall experience. These questionnaires were answer on a five point Likert scale. Moreover, participants were asked at the end about fatigue and preference between the different interfaces. For the results and analysis, the authors the Kruskal-Wallis rank sum test \cite{theodorsson1986kruskal} to understand the significance of the comparable conditions. Where if a comparison result is taken as representative, researchers should employ perspective cues with caution. Finally, for a user opinion study, the authors used the Wilcoxon test \cite{wilcoxon1945individual}, a pairwise ANOVA \cite{hoaglin1978hat}, to indicate design combinations and to found statistical significance under user opinions.

\clearpage

At the end, the authors concluded that performance was not significantly affected by the illusion of depth, when tested under directly comparable conditions. However, the inclusion of a certain background cues can have a significantly detrimental effect on performance. That said, it is important to take conclusions into consideration for our work. While medical images have typically a black background and are surrounded by the viewport tools and features. Also typically black. Changes of the design could, in fact, rise to significant improvement in recognition performance. There is evidence that combinations of visual depth cues substantially and significantly reinforces the visual effect. Based on this evidence, authors encourage the exploration of a larger design alternatives and applications to further research of the effects of these alternatives. Therefore, it is an interesting proposal to test several alternatives on a Medical Imaging application by testing task performance and radiologist's workflow changes.