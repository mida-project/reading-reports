\section{Final Evaluation}

%We have presented several research work from several authors. From Information Visualisation (InfoVis) experts, to Machine Learning (ML) ones, we found a large contribution to our work as described above. Based on their result evidence, we are now encourage to explore their work at a more deeper way by using their alternative methods for a further research of the effects of these alternatives on our system and interaction techniques.

Human-Computer Interaction (HCI) and User-Centered Design (UCD) are both fastest growing field in Computer Science (CS), and Health Informatics (HI) is amongst the greatest application challenges, providing future benefits in improved medical diagnosis, disease analyses, and health development. However, successful HCI and UCD for HI needs a concerted effort, fostering integrative research between experts ranging from diverse disciplines from User Research (UR) to Information Visualization (InfoVis).

Recommendation systems are an example for applications providing predictors, however, they have not yet been used extensively in HI and medical scenarios. This is due to the specifics of conservative criteria in medical scenarios and the multitude of drastically differing clinician groups and the enormous context complexity of the medical domain. Both, are risk perceptions towards quantitative and qualitative medical data as well as trust in accurate systems play a specific and central role, particularly in the clinical context. These aspects dominate acceptance of such systems. By using a Clinician-In-The-Loop (CITL) approach where several difficulties could be mitigated by combining both human expertise with computer efficiency. The work regarding \textbf{Classifier Predictors} from Ribeiro et al. \cite{ribeiro2016should} showed us this point of view, explaining the application of NNs over HI.